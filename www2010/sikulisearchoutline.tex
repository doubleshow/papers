\section{Introduction}

A: Getting help is important. Going to web for help. Web has many
resources, such as online forums, tutorials, official guides,
books. Content of reference books are also available online.

B: However, current methods for searching this resource has many
limitations. Because
* Need to type many keywords. Most cases copy and paste is not supported.
* Ambiguity: how to do X with Y? But keywords are all the same.
* Hard to judge the relevancy.
* Sometimes in a long document with multiple steps, it's unclear
where in the document is relevant to the current screen.
* can not search cross langauge (bi-lingual who may be familiar
with the terms in one language but can read the explanation in two
languages equally well)

C: Use multi-modal search
    Brief system demo
        walk through a search scenario
        showoff the scale
    Explain how each limitation is overcome
    List of contributions
        Propose a new large-scale search system
        Evaluate it
    Road map

\section{Related Work}
    WWW citations about image search
    Vision citations

\section{Challenges}

\section{Building the database}

\subsection{Collecting images}

With lots of resources like a commercial search companies, it is
possible to crawl the web and collect a large number of
screenshots. As an academic research project, we are limited by
resource. Yet, we still manage to build a research prototype
database at a non-trivial scale of 100,000 images. These images
came from three sources.

First, we used commercial image search engines such as Bing. We
collected a set of keywords that tend to be associated with
interactive dialog windows such as office, xp, windows, vista,
mac, options, settings, network, preferences ... etc. We used
different combinations of these keywords as search terms for image
search.

While in most image search applications false positive in search
results may be a curse, it is a blessing for data collection
purpose. For example, using system preferences as search terms,
while more than half of images are indeed screenshots of the
system preferences window, many images are of other interactive
windows that happen to be on pages that contain either the keyword
system or the keyword preference. Nevertheless, these screen
images are useful for users who happen to wish to query the system
about those interactive windows.

The API has restriction to only the top 1000 results. To increase
the likelihood of having a screenshot image, we use the filter
function to restrict to only illustration and graphics (namely,
ignoring natural photographs). In most cases, about 80\% are left.
In theory it is possible to greatly expand the database if we have
inside access to the index of the commercial search engines. We
obtained a total about 50,000 images.


Second, we used reverse image search engine such as TinEye. We
manually captured screenshots of about 500 interactive windows
across the three most common OS platforms (XP, Vita, and Mac OS).
We obtained about 10,000 images in this way, which is about 10
results for each photo. TinEye is capable for returning matches
that are almost visually identical to the search image.

Third, we collected a library of electronic books on software
applications. We extracted all the figures embedded in the file.
We obtained about 40,000 images.

Each of the three sources have pros and cons. Images from image
search as associated with only by keywords and may contain many
screenshots that are not visually relevant at all. TinEye are
ranked purely by visual similarities; top pages may not
necessarily contain any useful information. Images from book
library are usually accompanied by quality text, but their scope
is very limited; they cover mostly popular software programs there
is financial incentives to publish books on. Our hope is that by
combining these three data sources, we can have a rich variety of
useful data both visually and textually relevant to user queries.


\subsection{Indexing images}
        use hamming embedding
        use ransac

\subsection{Extracting Text}

A table summarize what features are applicable to which data type.

Give a real example of each type of text.

\textbf{Title} We also extract the text in the Title tag from the
HTML header. For book, the title can be too general.

\textbf{Heading} Some web page may contain multiple figures
illustrating different parts and provide heading for each part.
This heading can be more specific than the page title. However, to
extracting section headings reliably is harder for books in pdf
format. Some PDFs have outlines. We can extract the nearest
chapter or section heading.

\textbf{Alt Text} ALT tag provides information redundancy in case
when the image fails to be displayed or for users with visual
impairment. Also, it is useful for keyword-based image search. ALT
tag text can be used to determine the content of the image,
allowing search to go from keyword to image. However, in our
application, such text is redundant. It does not provide any
additional information other than what the user already knows by
reading the figure. For example, the ALT tag for the System
Preferences window is System Preferences but it is already obvious
by reading the title of the window.

\textbf{Caption} In books, figures are often clearly labeled with
captions. Thus, the text in the captions are extracted. However,
there's less common to add captions on web images, other than
online photo albums.

\textbf{Explicit reference} A figure and references made to it may
not be on the same page. It is necessary to search a couple pages
before and after the page the figure is on. This is different from
web pages. On a web page, a figure and the text about it often
appears on the same page. In contrast, in a typical book page, it
is not uncommon that a paragraph may refer to a figure one or two
pages ahead or after the page the paragraph is on due to
formatting reason. For example, a paragraph on page 100 may say
\emph{as shown in Figure 12-3} while the figure may be placed on
page 101 because of formatting constraints. Also, a paragraph can
make references such as \emph{as shown in Figure 12-3}. The
paragraph containing such explicit references can be useful too.


\textbf{Implicit references} There are also implicit references
that can be mined. For example, sentences like \emph{as shown
above} or \emph{see below} indicate that these sentences must be
relevant to the figure above or below respectively. These
sentences may not be immediately below or above the figure. But
one thing we know is that they must be not too far from the
figure. It is unlikely a paragraph may refer to a figure more than
10 pages away.

\textbf{Nearby text} If all above fails, we simply extract words
immediately above and below the image. On web page, images and
referring text tend to be on the same page. Also, relevant text
tends to be very near the image added for illustration. Typically,
the image is placed either above or below the text. Thus, we
extract the text immediately before and after the image tag.

\textbf{OCR} Useful text are those the user does not already know.
Text that can also be found in the figure is not as useful. Thus,
we apply OCR on each figure to determine what text is already
embedded in each figure. While these OCR text might not be that
useful to the users, since they can already read them, such text
can be useful for filtering out sentences that are simply repeat
of what is in the figure and thus less useful to the users.

\textbf{Action Sentences} With interactive applications, users
often care about what they can do with this interface. Because of
the limit in screen real estate, it is often impossible to put in
lots of text and caption to explain in details what each feature
can accomplish. Such details can be found on web page. Thus, we
focus on action keywords such as enable, allow, let, and extract
sentences containing these action keywords near the figure.




\section{Searching the database}

\subsection{Specifying queries}

There are two ways to specify image queries based on what the user
is seeing.

First, we provide a Java-based cross-platform search client
program. Users can download and run the program on any platform.
They can select any region on the screen by stretching a
rectangular box and submit the screen to the server. After
receiving the response, the program will automatically load the
default web browser to display the search result.

Second, users can also use our system without installing any
special software. Users can use the built-in image capture
utility, such as the snipping tool on Windows Vista or the
Command-Shift+4 hotkey on Mac OS. After capturing a screenshot
image, the user can use the submit form to send the query image to
the server and see the results.

Keywords are optional in our search system. They can be specified
in two ways. After capturing the image users can add a few
keywords in both the client program and the online form.
Alternatively, users can type keywords in the search box on the
result page.

Keywords can serve two purposes. First, they can specify the
content users wish to see on the returned pages. Second, they can
describe the screenshot to improve the performance of visual
matching. Since the screenshot matching can be done very
accurately, we expect users will very quickly realize that it is
not necessary to repeat the text embedded in the screenshot, since
the visual search will take care of it.


%Type keywords to filter the results. Keywords can be entered
%either together when the screenshot is captured, or after the
%results are displayed.

\subsection{Finding similar images}

%\subsection{Finding relevant pages by keywords]
% merged into Features

\subsection{Ranking pages}

Note that our results are pages not just images. In other words,
we are ranking pages where visual similarity of the embedded
figures plays a very critical role.

Results should be ranked by overall relevancy. Overall relevancy
can depend on the following features.

\subsubsection{Features}

Visual similarity: if the image is not similar at all, it's
unlikely the page text would contain anything useful

Image dimensions: If the image is too small, it may be a thumbnail
and may not be that useful.

Amount of text: the more text the page has, the more information
this page may contain.

Number of other images on the page: the fewer other images a page
has, the more likely the information on the page will be about the
image.

Image position: if the image occupies the prominent position on
the page, it's more likely to be something important.

Host authority: If an image is from Microsoft, it's more likely to
contain useful information. Number of other images on the site: if
a site hosts a lot of screenshot images, it's probably hosting the
images for some good purposes and those images are more likely to
be useful.

Quality of other images on the site: If screenshot images hosted
by a site are known to be accompanied by high-quality text, it's
more likely new screen images found on this site will also be
accompanied by high-quality text.

Keyword relevancy: if the page contains more keywords relevant to
those supplied by the users, the page should ranked higher

\subsubsection{Learning weights}

\subsection{Presenting the results}

New presentation scheme is needed for our purpose. In text search,
the keywords are highlighted, and displayed in the context of the
text before and after them. In image search, only the caption,
url, and dimensions are displayed. Users pay attention to images
to judge visual relevancy. Also, they pay attention to the url to
judge authority. In Tineye search (reverse image search), the
matched images are shown as thumbnails, grouped by "exact copies",
and ordered by visual similarity. The back links to the pages
where matched images are found are also displayed. All these are
not suitable for our purpose.





\section{Evaluation}

\subsection{Comparison to keyword search}

\subsection{Comparison of several ranking strategies}

\section{Summary}
